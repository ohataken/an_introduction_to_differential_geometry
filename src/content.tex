\title{\huge 微分幾何入門}
\author{ohataken}
\maketitle

\newpage

% ================================
% ================================
% ================================

\section{いろいろな距離}

多様体を理解するための、できるだけ直感的なアプローチを考えてみると、やはり距離からはじめるのがよいと思う。小学校では「たかしくんは家から学校まで〜」のような問題があったし、高校では四面体の頂点と底面の距離をベクトルを使って求めてきた。また、数学から離れた素朴な感覚として、街を歩くときも、電車に乗るときも、車を運転するときも、距離を意識していると思う。人間の素朴な距離の感覚を否定せず、数学の厳密さと整合するようにていねいに作られたのが多様体なのだと思う。

距離とは、よく知られているユークリッド距離だけが唯一の距離公式だというわけではなく、実はいくつもの距離公式がある。いろいろな距離を並べてみることで、どういう共通点があるのか、どういう定義をするとそれらをうまく説明できるか、という抽象化のモチヴェーションが生まれる。その先に多様体がある。

\newpage

% ================================

\subsection{ユークリッド距離}

ユークリッド距離

\newpage

% ================================

\subsection{マンハッタン距離}

ただ縦と横を足すだけの、自乗とかルートとか使わない距離は、マンハッタン距離と呼ばれている。マンハッタンは方眼紙のような区画になっていて、斜めに移動することができず、単純に縦と横の足したものとして考えてよいらしい。日本人としては京都距離とでも呼びたいところだが、この名で知られている。

\newpage

% ================================

\subsection{フランス鉄道距離}

フランス鉄道距離

\newpage

% ================================

\subsection{球面の距離}

球面の距離

\newpage

% ================================

\subsection{楕円幾何の距離}

楕円幾何の距離

\newpage

% ================================

\subsection{双曲幾何 上半平面モデルの距離}

双曲幾何 上半平面モデルの距離

\newpage

% ================================

\subsection{双曲幾何 ポアンカレモデルの距離}

双曲幾何 ポアンカレモデルの距離

\newpage

% ================================
% ================================
% ================================

\section{距離のしくみ}

距離のしくみ

\newpage

% ================================

\subsection{距離の公理}

距離の公理

\newpage

% ================================

\subsection{開集合系}

開集合系

\newpage

% ================================

\subsection{ユークリッド位相}

ユークリッド位相

\newpage

% ================================

\subsection{離散位相}

離散位相

\newpage

% ================================

\subsection{密着位相}

密着位相

\newpage

% ================================

\subsection{位相空間}

位相空間

\newpage

% ================================
% ================================
% ================================

\section{計量}

計量

\newpage
