\title{ \huge 微分幾何入門 }
\author{ ohataken }
\maketitle

\newpage

% ================================
% ================================
% ================================

\section{ いろいろな距離 }

多様体を理解するための、できるだけ直感的なアプローチを考えてみると、やはり距離からはじめるのがよいと思う。小学校では「たかしくんは家から学校まで〜」のような問題があったし、高校では四面体の頂点と底面の距離をベクトルを使って求めてきた。また、数学から離れた素朴な感覚として、街を歩くときも、電車に乗るときも、車を運転するときも、距離を意識していると思う。人間の素朴な距離の感覚を否定せず、数学の厳密さと整合するようにていねいに作られたのが多様体なのだと思う。

距離とは、よく知られているユークリッド距離だけが唯一の距離公式だというわけではなく、実はいくつもの距離公式がある。いろいろな距離を並べてみることで、どういう共通点があるのか、どういう定義をするとそれらをうまく説明できるか、という抽象化のモチヴェーションが生まれる。その先に多様体がある。

\newpage

% ================================

\subsection{ ユークリッド距離 }

ユークリッド距離

\newpage

% ================================

\subsection{ 球面の距離 }

球面の距離

\newpage

% ================================

\subsection{ 楕円幾何の距離 }

楕円幾何の距離

\newpage

% ================================

\subsection{ 双曲幾何 上半平面モデルの距離 }

双曲幾何 上半平面モデルの距離

\newpage

% ================================

\subsection{ 双曲幾何 開円板モデルの距離 }

双曲幾何 開円板モデルの距離

\newpage

% ================================
% ================================
% ================================

\section{ 道のり }

\newpage

\subsection{ ユークリッド距離を使った道のり }

ユークリッド距離を使った道のり

\newpage

% ================================

\subsection{ 球面における道のり }

球面における道のり

\newpage

% ================================

\subsection{ 楕円幾何における道のり }

楕円幾何における道のり

\newpage

% ================================

\subsection{ 双曲幾何 上半平面モデルにおける道のり }

双曲幾何 上半平面モデルにおける道のり

\newpage

% ================================

\subsection{ 双曲幾何 開円板モデルにおける道のり }

双曲幾何 開円板モデルにおける道のり

\newpage

% ================================
% ================================
% ================================

\section{ 計量 }

ここまで、いくつかの距離や道のりの例を見てきた。ここでは、少し視座を高めて、それらを抽象化した計量というものを考える。

計量を使って考えると、ユークリッド距離なのか、あるいはポアンカレの開円板モデルの距離なのか、といった具体的な距離が何か?ということが枝葉末節になる。

抽象化といっても、かなり自然なもので、いままで意識していなかったが、距離を計算するときに、見えない行列がかかっていたのだ、と考える。

\newpage

% ================================
% ================================
% ================================

\section{ 測地線 }

測地線

\newpage

% ================================
% ================================
% ================================

\section{ 相対性理論 }

相対性理論

\newpage

% ================================
% ================================
% ================================

\section{ 電磁気学とベクトル場 }

電磁気学とベクトル場

\newpage
